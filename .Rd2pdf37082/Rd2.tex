\documentclass[letterpaper]{book}
\usepackage[times,hyper]{Rd}
\usepackage{makeidx}
\usepackage[utf8]{inputenc} % @SET ENCODING@
% \usepackage{graphicx} % @USE GRAPHICX@
\makeindex{}
\begin{document}
\chapter*{}
\begin{center}
{\textbf{\huge httrpathway}}
\par\bigskip{\large \today}
\end{center}
\begin{description}
\raggedright{}
\inputencoding{utf8}
\item[Type]\AsIs{Package}
\item[Title]\AsIs{Pathway Scoring and Concentration Response for HTTr data}
\item[Version]\AsIs{1.1.0}
\item[Author]\AsIs{Thomas Sheffield}
\item[Maintainer]\AsIs{Richard Judson }\email{judson.richard@epa.gov}\AsIs{}
\item[Description]\AsIs{This package generates pathway (signature) scores with associated concentration response
modeling; it also contains some important plotting functions. This package contains functions
required to create input files (log2-fold change, or (l2fc) matrices) and run the signature/pathway
based concentration-response calculations. Another R project (httranalysis) contains a series of post-calculation
analyses that are problem-specific. To run all of the calculations, use the function driver().
This version has also included gene-level concentration-response modeling
This package required a set of directories to be at the same level as the httrpathway folder
../input - various input files
../input/chemicals - collections of chemical information, not used in the standard calculations
../input/signatures - the signature data, inluding the catalog (an Excel file) and the lists of genes per signature
../input/fcdata - where the l2fc data goes. See the functions buildFCMAT1 and buildFCMAT2 for more information. These functions may need to be customizedfor the source of your data
../output - where all of the output goes [not clear if the subfolders are created on demand]
There are a series of data sets / objects that are names and carried around:
* dataset - this is the name of the data set being used. It corresponds to an experiment and the name ususally contains the cell type, the type of normalization, the time, media, etc. All input and output files will contain this dataset name
* sigcatalog - This is the name of the signature catalog. This is an excel file that lives in ../input/signatures.This file contains one row per signature and contains matching annotations such as the super\_target
* sigset - One always uses a subset of the total set of signtures, indicated by haveing a value of 1 in the sigset column at the right hand of the signature catalog}
\item[Imports]\AsIs{stats,
stringr,
grDevices,
graphics,
utils,
methods,
data.table,
future.apply,
future,
GSVA,
moments,
numDeriv,
openxlsx,
parallel,
RColorBrewer,
reshape2,
data.table,
openxlsx,
e1071,
tidyverse}
\item[License]\AsIs{MIT + file LICENSE}
\item[Encoding]\AsIs{UTF-8}
\item[LazyData]\AsIs{true}
\item[RoxygenNote]\AsIs{7.1.1}
\item[Suggests]\AsIs{knitr,
rmarkdown}
\item[VignetteBuilder]\AsIs{knitr}
\end{description}
\Rdcontents{\R{} topics documented:}
\inputencoding{utf8}
\HeaderA{auc}{Area Under the Curve}{auc}
%
\begin{Description}\relax
Compute AUC for an ROC curve.
\end{Description}
%
\begin{Usage}
\begin{verbatim}
auc(tpr, fpr)
\end{verbatim}
\end{Usage}
%
\begin{Arguments}
\begin{ldescription}
\item[\code{tpr}] Vector of true positive rates.

\item[\code{fpr}] Vector of false positive rates.
\end{ldescription}
\end{Arguments}
%
\begin{Details}\relax
Uses trapezoid rule numerical integration to approximate AUC. Will be more
accurate with more fine-grained inputs.
\end{Details}
%
\begin{Value}
AUC
\end{Value}
%
\begin{Examples}
\begin{ExampleCode}
auc(c(0,.5,1), c(0,.5,1))
auc(c(0,1,1), c(0,.5,1))
\end{ExampleCode}
\end{Examples}
\inputencoding{utf8}
\HeaderA{baseline\_gene\_counts}{Gene the baseline gene counts for the cell atlas project}{baseline.Rul.gene.Rul.counts}
%
\begin{Description}\relax
Gene the baseline gene counts for the cell atlas project
\end{Description}
%
\begin{Usage}
\begin{verbatim}
baseline_gene_counts(
  db = "httr_cell_atlas",
  dir = "../input/rawdata/cellatlas/"
)
\end{verbatim}
\end{Usage}
%
\begin{Arguments}
\begin{ldescription}
\item[\code{db}] The name of the Mongo database

\item[\code{dir}] The directory where the data will be stored

This functions takes files created by export\_mongo\_httr\_well()
* httr\_cell\_atlas
* httr\_tox21\_cpp2
\end{ldescription}
\end{Arguments}
\inputencoding{utf8}
\HeaderA{bioplanet\_builder}{BioPlanet Builder}{bioplanet.Rul.builder}
%
\begin{Description}\relax
Converts BioPlanet data into usable pathway data.
\end{Description}
%
\begin{Usage}
\begin{verbatim}
bioplanet_builder(
  pathfile = "../input/processed_pathway_data/bioplanet_pathway.csv",
  catfile = "../input/processed_pathway_data/bioplanet_pathway_category.csv",
  pwayout = "../input/processed_pathway_data/bioplanet_PATHWAYS.RData",
  pdataout = "../input/processed_pathway_data/PATHWAY_LIST_bioplanet.RData"
)
\end{verbatim}
\end{Usage}
%
\begin{Arguments}
\begin{ldescription}
\item[\code{pathfile}] File name of bioplanet\_pathway.csv.

\item[\code{catfile}] File name of bioplanet\_pathway\_category.csv.

\item[\code{pwayout}] File name of bioplanet\_PATHWAYS.RData

\item[\code{pdataout}] File name of
\end{ldescription}
\end{Arguments}
%
\begin{Details}\relax
This function shows how BioPlanet data was converted to usable pathway files.
As BioPlanet is updated, this function will have to be updated. It requires
two downloaded .csv files with location specified by pathfile and catfile.
It saves usable pathway files with location specified by pwayout and
pdataout to disk.
\end{Details}
%
\begin{Value}
No output.
\end{Value}
\inputencoding{utf8}
\HeaderA{buildFCMAT1.fromDB}{Build the FCMAT1 data set}{buildFCMAT1.fromDB}
%
\begin{Description}\relax
version to start with Logan's database export
The difference between this version and the original is that there are extra columns
The function just changes one column name and writes the file to a standard name and place
\end{Description}
%
\begin{Usage}
\begin{verbatim}
buildFCMAT1.fromDB(
  dataset = "tox21_cpp5_u2os_pe1_normal",
  dir = "../input/fcdata/new_versions/",
 
    infile = "httr_tox21_cpp5_u2os_FCmat1-meanncnt0_5-plateteffect_1-shrinkage_normal.RData",
  pg.filter.file = NULL,
  do.load = T
)
\end{verbatim}
\end{Usage}
%
\begin{Arguments}
\begin{ldescription}
\item[\code{dataset}] The name to give to the data set

\item[\code{dir}] The directory from which to read all of the raw files

\item[\code{infile}] The nae of the input file

\item[\code{pg.filter.file}] An optional file to use in filtering out bad plate groups

\item[\code{do.load}] If TRUE, read the large input data file into memory
\end{ldescription}
\end{Arguments}
%
\begin{Value}
A file with the FCMAT1 data is written to "../input/fcdata/FCMAT1\_",dataset,".RData"
\end{Value}
\inputencoding{utf8}
\HeaderA{buildFCMAT2.fromDB}{Transpose and filter the fold change matrix FCMAT1 in long format into a gene x sample format.}{buildFCMAT2.fromDB}
%
\begin{Description}\relax
Transpose and filter the fold change matrix FCMAT1 in long format into a
gene x sample format.
\end{Description}
%
\begin{Usage}
\begin{verbatim}
buildFCMAT2.fromDB(
  dataset = "tox21_cpp5_u2os_pe1_normal",
  time = 24,
  media = "DMEM",
  dir = "../input/fcdata/",
  method = "gene",
  do.read = T
)
\end{verbatim}
\end{Usage}
%
\begin{Arguments}
\begin{ldescription}
\item[\code{dataset}] The name to give to the data set

\item[\code{time}] The time in hours that the chemical dosing was run

\item[\code{media}] THe name of the media used

\item[\code{dir}] The directory from which to read all of the raw files

\item[\code{method}] Either "gene" or "probe"

\item[\code{do.read}] If TRUE, read in the FCMAT1 file and place in a global.
\end{ldescription}
\end{Arguments}
%
\begin{Value}
Global variables are created for the FC matrix (FCMAT2), the SE matrix (SEMAT2)
and the chemical dictionary (CHEM\_DICT) which translates form the sample key
(sample\_id\_conc\_time) to the individual components
\end{Value}
\inputencoding{utf8}
\HeaderA{buildFCMAT2.fromDB.refchems}{Transpose and filter the fold change matrix FCMAT1 in long format into a gene x sample format. This is the method to use when there are conc-response profiles of refchems}{buildFCMAT2.fromDB.refchems}
%
\begin{Description}\relax
Transpose and filter the fold change matrix FCMAT1 in long format into a
gene x sample format.
This is the method to use when there are conc-response profiles of refchems
\end{Description}
%
\begin{Usage}
\begin{verbatim}
buildFCMAT2.fromDB.refchems(
  dataset = "heparg2d_toxcast_pfas_pe1_normal_v2",
  time = 24,
  media = "DMEM",
  dir = "../input/fcdata/",
  method = "gene",
  do.read = F,
  do.prep = T
)
\end{verbatim}
\end{Usage}
%
\begin{Arguments}
\begin{ldescription}
\item[\code{dataset}] The name to give to the data set

\item[\code{time}] The time in hours that the chemical dosing was run

\item[\code{media}] THe name of the media used

\item[\code{dir}] The directory from which to read all of the raw files

\item[\code{method}] Either "gene" or "probe"

\item[\code{do.read}] If TRUE, read in the FCMAT1 file and place in a global.
\end{ldescription}
\end{Arguments}
%
\begin{Value}
Global variables are created for the FC matrix (FCMAT2), the SE matrix (SEMAT2)
and the chemical dictionary (CHEM\_DICT) which translates form the sample key
(sample\_id\_conc\_time) to the individual components
\end{Value}
\inputencoding{utf8}
\HeaderA{buildSampleMap}{Generate the sample\_key x sample x DSSTox file}{buildSampleMap}
%
\begin{Description}\relax
Generate the sample\_key x sample x DSSTox file
\end{Description}
%
\begin{Usage}
\begin{verbatim}
buildSampleMap(
  dataset = "DMEM_6hr_pilot_normal_pe_1",
  dsstox.file = "../input/DSSTox/DSSTox_sample_map.xlsx",
  dir = "../input/fcdata/",
  outfile = "../input/chemicals/HTTr_pilot_sample_map.xlsx",
  do.read = F
)
\end{verbatim}
\end{Usage}
%
\begin{Arguments}
\begin{ldescription}
\item[\code{dataset}] Name of hte HTTr dataset

\item[\code{dsstox.file}] Name of the DSStox chemical file

\item[\code{dir}] Directory where the FCMAT1 files lives

\item[\code{outfile}] Name of the output file

\item[\code{do.read}] If TRUE, read in the input FCMAT1 file
\end{ldescription}
\end{Arguments}
\inputencoding{utf8}
\HeaderA{buildStudyChemicalMap}{Build a catalog of the chemicals in a dataset}{buildStudyChemicalMap}
%
\begin{Description}\relax
Build a catalog of the chemicals in a dataset
\end{Description}
%
\begin{Usage}
\begin{verbatim}
buildStudyChemicalMap(dataset = "DMEM_6hr_screen_normal_pe_1")
\end{verbatim}
\end{Usage}
%
\begin{Arguments}
\begin{ldescription}
\item[\code{dataset}] The name of the HTTr dataset
\end{ldescription}
\end{Arguments}
%
\begin{Value}
No output.
\end{Value}
\inputencoding{utf8}
\HeaderA{calcDEG}{Calculate the relative variability of genes to get the DEGs}{calcDEG}
%
\begin{Description}\relax
Calculate the relative variability of genes to get the DEGs
\end{Description}
%
\begin{Usage}
\begin{verbatim}
calcDEG(
  dataset = "mcf7_ph1_pe1_normal_good_pg",
  dir = "../input/fcdata/",
  do.read = T
)
\end{verbatim}
\end{Usage}
%
\begin{Arguments}
\begin{ldescription}
\item[\code{dataset}] The name to give to the data set

\item[\code{dir}] The directory from which to read all of the raw filesatalog file

\item[\code{do.read}] If TRUE, read in the HTTr data file
\end{ldescription}
\end{Arguments}
\inputencoding{utf8}
\HeaderA{concatDESeq2Files}{Concatenate the input DESeq2 files}{concatDESeq2Files}
%
\begin{Description}\relax
Concatenate the input DESeq2 files
\end{Description}
%
\begin{Usage}
\begin{verbatim}
concatDESeq2Files(
  dataset = "DMEM_6hr_screen_normal_pe_1",
 
    indir = "../input/httr_mcf7_screen/meanncnt0_5-plateteffect_0-shrinkage_normal_DMEM_6/",
  outdir = "../input/httr_mcf7_screen/"
)
\end{verbatim}
\end{Usage}
%
\begin{Arguments}
\begin{ldescription}
\item[\code{dataset}] The name of the HTTr dataset

\item[\code{indir}] The director to read from

\item[\code{outdir}] The directory to write to
\end{ldescription}
\end{Arguments}
\inputencoding{utf8}
\HeaderA{cutoffCalc}{Calculate the signature-wise cutoffs based on the analytical method which does not break any correlations between genes}{cutoffCalc}
%
\begin{Description}\relax
Calculate the signature-wise cutoffs based on the analytical method
which does not break any correlations between genes
\end{Description}
%
\begin{Usage}
\begin{verbatim}
cutoffCalc(
  basedir = "../input/fcdata/",
  dataset,
  sigcatalog,
  sigset,
  method,
  pval = 0.05,
  seed = 12345,
  nlowconc = 2,
  mc.cores = 1,
  dtxsid.exclude = NULL,
  do.load = T,
  do.cov = T,
  do.compare = F,
  to.file = F,
  verbose = F
)
\end{verbatim}
\end{Usage}
%
\begin{Arguments}
\begin{ldescription}
\item[\code{basedir}] Directory that holds FCMAT2 and CHEM\_DICT files.

\item[\code{dataset}] Name of actual dataset to base cutoff on.

\item[\code{sigcatalog}] The name of the signature catalog to use

\item[\code{sigset}] THe signature set

\item[\code{method}] The scoring method, either fc or gsea

\item[\code{pval}] The p-value for the baseline distribution

\item[\code{seed}] Random seed.

\item[\code{nlowconc}] Only include the lowest nlowconc concentrations for each chemical

\item[\code{mc.cores}] NUmber of coresto use when running parallel

\item[\code{dtxsid.exclude}] dtxsids to exclude, default NULL

\item[\code{do.load}] If TRUE, reload the FCMAT2 matrix, signature catalog and chemical dictionary, and store in globals

\item[\code{do.cov}] If TRUE, calculate the covariance matrix and store in a global

\item[\code{do.compare}] If TRUE, compare the cutoffs with those from the original method with no gene-gene correlation

\item[\code{to.file}] If TRUE, and do.compare=TRUE, send a plot of the comparison to a file

\item[\code{verbose}] If TRUE, write a line for each signature to show progress.
\end{ldescription}
\end{Arguments}
%
\begin{Value}
No output.
\end{Value}
\inputencoding{utf8}
\HeaderA{cutoffCalc.inner.emperical}{Inner function for the cutoff calculation}{cutoffCalc.inner.emperical}
%
\begin{Description}\relax
Inner function for the cutoff calculation
\end{Description}
%
\begin{Usage}
\begin{verbatim}
cutoffCalc.inner.emperical(signature, pval)
\end{verbatim}
\end{Usage}
%
\begin{Arguments}
\begin{ldescription}
\item[\code{signature}] The name of the signature  for which the cutoff is to be calculated

\item[\code{pval}] The p-value for the baseline distribution

\item[\code{covmat}] THe covariance matrix
\end{ldescription}
\end{Arguments}
%
\begin{Value}
vector containing the signature, cutoff, sd, bmed
\end{Value}
\inputencoding{utf8}
\HeaderA{cutoffCalc.inner.fc}{Inner function for the cutoff calculation}{cutoffCalc.inner.fc}
%
\begin{Description}\relax
Inner function for the cutoff calculation
\end{Description}
%
\begin{Usage}
\begin{verbatim}
cutoffCalc.inner.fc(parent, catalog, allgenes, pval)
\end{verbatim}
\end{Usage}
%
\begin{Arguments}
\begin{ldescription}
\item[\code{parent}] The name of the signature parent for which the cutoff is to be calculated

\item[\code{catalog}] The signature catalog

\item[\code{allgenes}] THe list of all the genes in the data set

\item[\code{pval}] The p-value for the baseline distribution

\item[\code{covmat}] THe covariance matrix
\end{ldescription}
\end{Arguments}
%
\begin{Value}
vector containing the parent (signature), cutoff, sd, bmed
\end{Value}
\inputencoding{utf8}
\HeaderA{cutoffCalcEmpirical}{Calculate the signature-wise cutoffs based on the empirical distributions which does not break any correlations between genes}{cutoffCalcEmpirical}
%
\begin{Description}\relax
Calculate the signature-wise cutoffs based on the empirical distributions
which does not break any correlations between genes
\end{Description}
%
\begin{Usage}
\begin{verbatim}
cutoffCalcEmpirical(
  basedir = "../input/fcdata/",
  dataset = "heparg2d_toxcast_pfas_pe1_normal",
  sigset = "screen_large",
  method = "fc",
  pval = 0.05,
  nlowconc = 2,
  mc.cores = 1,
  dtxsid.exclude = NULL,
  do.load = T
)
\end{verbatim}
\end{Usage}
%
\begin{Arguments}
\begin{ldescription}
\item[\code{basedir}] Directory that holds FCMAT2 and CHEM\_DICT files.

\item[\code{dataset}] Name of actual dataset to base cutoff on.

\item[\code{sigset}] THe signature set

\item[\code{method}] The scoring method, either fc or gsea

\item[\code{pval}] The p-value for the baseline distribution

\item[\code{nlowconc}] Only include the lowest nlowconc concentrations for each chemical

\item[\code{mc.cores}] NUmber of cores to use when running parallel

\item[\code{dtxsid.exclude}] dtxsids to exclude, default NULL

\item[\code{do.load}] If TRUE, reload the FCMAT2 matrix, signature catalog and chemical dictionary, and store in globals

\item[\code{sigcatalog}] The name of the signature catalog to use

\item[\code{do.cov}] If TRUE, calculate the covariance matrix and store in a global

\item[\code{do.compare}] If TRUE, compare the cutoffs with those from the original method with no gene-gene correlation

\item[\code{to.file}] If TRUE, and do.compare=TRUE, send a plot of the comparison to a file

\item[\code{verbose}] If TRUE, write a line for each signature to show progress.
\end{ldescription}
\end{Arguments}
%
\begin{Value}
No output.
\end{Value}
\inputencoding{utf8}
\HeaderA{cutoffPlot}{Calculate the signature-wise cutoffs based on the analytical method which does not break any correlations between genes}{cutoffPlot}
%
\begin{Description}\relax
Calculate the signature-wise cutoffs based on the analytical method
which does not break any correlations between genes
\end{Description}
%
\begin{Usage}
\begin{verbatim}
cutoffPlot(
  to.file = F,
  dataset = "mcf7_ph1_pe1_normal_block_123_allPG",
  sigset = "screen_large",
  method = "fc",
  pval = 0.05,
  nlowconc = 2
)
\end{verbatim}
\end{Usage}
%
\begin{Arguments}
\begin{ldescription}
\item[\code{to.file}] If TRUE, and do.compare=TRUE, send a plot of the comparison to a file

\item[\code{dataset}] Name of actual dataset to base cutoff on.

\item[\code{sigset}] THe signature set

\item[\code{method}] The scoring method, either fc or gsea

\item[\code{pval}] The p-value for the baseline distribution

\item[\code{nlowconc}] Only include the lowest nlowconc concentrations for each chemical

\item[\code{basedir}] Directory that holds FCMAT2 and CHEM\_DICT files.

\item[\code{sigcatalog}] The name of the signature catalog to use

\item[\code{seed}] Random seed.

\item[\code{mc.cores}] NUmber of coresto use when running parallel

\item[\code{dtxsid.exclude}] dtxsids to exclude, default NULL

\item[\code{do.load}] If TRUE, reload the FCMAT2 matrix, signature catalog and chemical dictionary, and store in globals

\item[\code{do.cov}] If TRUE, calculate the covariance matrix and store in a global

\item[\code{do.compare}] If TRUE, compare the cutoffs with those from the original method with no gene-gene correlation

\item[\code{verbose}] If TRUE, write a line for each signature to show progress.
\end{ldescription}
\end{Arguments}
%
\begin{Value}
No output.
\end{Value}
\inputencoding{utf8}
\HeaderA{driver}{Code to run all signature concentration-response calculations}{driver}
%
\begin{Usage}
\begin{verbatim}
driver(
  dataset = "mcf7_ph1_pe1_normal_block_123_allPG",
  sigcatalog = "signatureDB_master_catalog 2021-10-05 unidirectional",
  sigset = "screen_large_unidirectional",
  cutoff.dataset = NULL,
  normfactor = 7500,
  mc.cores = 10,
  bmr_scale = 1.349,
  pval = 0.05,
  nlowconc = 2,
  hccut = 0.9,
  tccut = 1,
  plotrange = c(1e-04, 100),
  method = "gsea",
  celltype = "MCF7",
  do.conc.resp = T,
  do.scr.plots = T,
  do.signature.pod = F,
  do.supertarget.boxplot = T,
  do.all = F
)
\end{verbatim}
\end{Usage}
%
\begin{Arguments}
\begin{ldescription}
\item[\code{dataset}] Name of the data set, produced by buildFCMAT2

\item[\code{sigcatalog}] Name of the signature catalog

\item[\code{sigset}] Name if the signature set. THis corresponds to a column in the signature catalog file

\item[\code{cutoff.dataset}] This is the data set name to sue when the cutoffs are taken from a different data set than
the one currently being analyzed. The reason for doing this is if the current data set is small
(small number of chemicals), and so not large enough to get a good estiamte of the underlying
noise distribution. All of the other parameters for both data sets have to be the same

\item[\code{normfactor}] Normalization factor for the conc-reap plots, default is 7500

\item[\code{mc.cores}] Number of cores for parallel processing. Only works under Linux

\item[\code{bmr\_scale}] Scaling factor from the NULL SD to BMD, default is 1.349

\item[\code{pval}] Threshold for cutoff distribution confidence interval. Default=0.05 indicates a 95

\bsl{}itemnlowconcOnly include the lowest nlowconc concentrations for each chemical

\bsl{}itemhccutThe threshold for signatures to be called a hit, default=0.95,

\bsl{}itemtccutThe threshold for top/cutoff o be a hit, default =1.5

\bsl{}itemplotrangeThe concentration range for the conc-resp plots in uM, default is c(0.0001,100),

\bsl{}itemmethodsignature scoring method in c("fc", "gsva", "gsea"), default is fc

\bsl{}itemcelltypeName of the cull type, e.g. MCF7

\bsl{}itemdo.conc.respIf true, run the concentration-response calculations

\bsl{}itemdo.scr.plotsIf TRUE, generate the signature concentration response plots

\bsl{}itemdo.signature.podIf TRUE, generate the signature PODs

\bsl{}itemdo.supertarget.boxplotIf TRUE, generate the super target box plots

\bsl{}itemdo.allIf TRUE, do all steps from do.build.random to the end


Available data sets
* heparg2d\_toxcast\_pfas\_pe1\_normal\_v2
* mcf7\_ph1\_pe1\_normal\_block\_123\_allPG
* u2os\_toxcast\_pfas\_pe1\_normal\_v2
* PFAS\_HepaRG
* PFAS\_U2OS
* u2os\_pilot\_pe1\_normal\_null\_pilot
* u2os\_toxcast\_pfas\_pe1\_normal\_v2\_refchems
* heparg2d\_toxcast\_pfas\_pe1\_normal\_v2\_refchems
* DMEM\_6hr\_pilot\_normal\_pe\_1 - MCF7 pilot

* MCF7\_pilot\_DMEM\_6hr\_pilot\_normal\_pe\_1
* MCF7\_pilot\_DMEM\_12hr\_pilot\_normal\_pe\_1
* MCF7\_pilot\_DMEM\_24hr\_pilot\_normal\_pe\_1
* MCF7\_pilot\_PRF\_6hr\_pilot\_normal\_pe\_1
* MCF7\_pilot\_PRF\_12hr\_pilot\_normal\_pe\_1
* MCF7\_pilot\_PRF\_24hr\_pilot\_normal\_pe\_1

* tox21\_cpp5\_u2os\_pe1\_normal
* tox21\_cpp5\_heparg\_pe1\_normal

\bsl{}itemdo.signature.summary.plotif TRUE, generate the summary plots

\bsl{}itemdo.signature.pod.laneplotIf TRUE, generate the signature lane plots (only useful for small sets of chemicals)


Code to run all signature concentration-response calculations

\end{ldescription}
\end{Arguments}
\inputencoding{utf8}
\HeaderA{exportDSSToxSample}{Generate the sample x DSSTox file}{exportDSSToxSample}
%
\begin{Description}\relax
Generate the sample x DSSTox file
\end{Description}
%
\begin{Usage}
\begin{verbatim}
exportDSSToxSample(outfile = "../input/DSSTox/DSSTox_sample_map.xlsx")
\end{verbatim}
\end{Usage}
%
\begin{Arguments}
\begin{ldescription}
\item[\code{outfile}] Name of the file to be written
\end{ldescription}
\end{Arguments}
\inputencoding{utf8}
\HeaderA{exportSignatureCutoffs}{Export the signature-wise cutoffs}{exportSignatureCutoffs}
%
\begin{Description}\relax
Export the signature-wise cutoffs
\end{Description}
%
\begin{Usage}
\begin{verbatim}
exportSignatureCutoffs(
  do.load = F,
  dataset = "heparg2d_toxcast_pfas_pe1_normal",
  sigset = "screen_large",
  method = "fc"
)
\end{verbatim}
\end{Usage}
%
\begin{Arguments}
\begin{ldescription}
\item[\code{do.load}] If TRUE, load hte large data file

\item[\code{dataset}] The name of the HTTr data set to use

\item[\code{sigset}] The name of the signature set to use

\item[\code{method}] The scoring method to use
\end{ldescription}
\end{Arguments}
\inputencoding{utf8}
\HeaderA{export\_mongo\_httr\_well}{Get the raw counts from the Mongo database}{export.Rul.mongo.Rul.httr.Rul.well}
%
\begin{Description}\relax
Get the raw counts from the Mongo database
\end{Description}
%
\begin{Usage}
\begin{verbatim}
export_mongo_httr_well(
  db = "httr_cell_atlas",
  collection = "httr_well_trt",
  dir = "../input/rawdata/cellatlas/"
)
\end{verbatim}
\end{Usage}
%
\begin{Arguments}
\begin{ldescription}
\item[\code{db}] The name of the Mongo database

\item[\code{collection}] THe name of the collection to export

\item[\code{dir}] The directory where the data will be stored

Collections
* httr\_cell\_atlas
* httr\_tox21\_cpp2
\end{ldescription}
\end{Arguments}
\inputencoding{utf8}
\HeaderA{fixSuperTarget}{Replace the super\_target values in the signature output file with ones from a new catalog}{fixSuperTarget}
%
\begin{Description}\relax
Replace the super\_target values in the signature output file
with ones from a new catalog
\end{Description}
%
\begin{Usage}
\begin{verbatim}
fixSuperTarget(
  do.read = T,
  dataset = "PFAS_U2OS",
  sigcatalog = "signatureDB_master_catalog 2021-05-10",
  sigset = "screen_large",
  method = "fc"
)
\end{verbatim}
\end{Usage}
%
\begin{Arguments}
\begin{ldescription}
\item[\code{do.read}] If TRUE, read in FCMAT2 to a global

\item[\code{dataset}] The L2fc matrix data set

\item[\code{sigcatalog}] THe name of the signature catalog file

\item[\code{sigset}] The name of the signature set to use

\item[\code{method}] The scoring method
\end{ldescription}
\end{Arguments}
\inputencoding{utf8}
\HeaderA{geneBaseMeanDist}{get the base mean distribution for each gene}{geneBaseMeanDist}
%
\begin{Description}\relax
get the base mean distribution for each gene
\end{Description}
%
\begin{Usage}
\begin{verbatim}
geneBaseMeanDist(
  to.file = F,
  do.read = F,
  dataset = "DMEM_6hr_screen_normal_pe_1"
)
\end{verbatim}
\end{Usage}
%
\begin{Arguments}
\begin{ldescription}
\item[\code{to.file}] If TRUE, plot to a file

\item[\code{do.read}] If TRUE, read the input file into memory

\item[\code{dataset}] The name of the dataset
\end{ldescription}
\end{Arguments}
%
\begin{Value}
No output.
\end{Value}
\inputencoding{utf8}
\HeaderA{geneConcResp}{Gene Concentration Response}{geneConcResp}
%
\begin{Description}\relax
Wrapper that performs concentration response modeling for gene or probe l2fc's
\end{Description}
%
\begin{Usage}
\begin{verbatim}
geneConcResp(
  dataset = "tox21_cpp5_heparg_pe1_normal",
  mc.cores = 20,
  to.file = T,
  pval = 0.05,
  aicc = F,
  fitmodels = c("cnst", "hill", "poly1", "poly2", "pow", "exp2", "exp3", "exp4",
    "exp5"),
  genefile = NULL
)
\end{verbatim}
\end{Usage}
%
\begin{Arguments}
\begin{ldescription}
\item[\code{dataset}] String that identifies data set.

\item[\code{mc.cores}] Number of parallel cores to use.

\item[\code{to.file}] If TRUE, results are written to an RData file, otherwise they
are returned.

\item[\code{pval}] P-value cutoff between 0 and 1.

\item[\code{aicc}] If aicc = T, corrected AIC is used insstead of first order
(regular) AIC.

\item[\code{fitmodels}] Vector of models names to be used. Default is all of them.
\end{ldescription}
\end{Arguments}
%
\begin{Details}\relax
Loads an FCMAT2 and CHEM\_DICT corresponding to given dataset. FCMAT should be
chem/conc by gene or chem/conc by probe. Uses two lowest concentration of each
column to estimate noise cutoff (as opposed to signature CR). Also, doesn't
currently contain a plotting option.
\end{Details}
%
\begin{Value}
If to.file = F, data frame containing results; otherwise, nothing.

* MCF7\_pilot\_DMEM\_6hr\_pilot\_normal\_pe\_1
* MCF7\_pilot\_DMEM\_12hr\_pilot\_normal\_pe\_1
* MCF7\_pilot\_DMEM\_24hr\_pilot\_normal\_pe\_1
* MCF7\_pilot\_PRF\_6hr\_pilot\_normal\_pe\_1
\end{Value}
\inputencoding{utf8}
\HeaderA{geneConcRespPlot}{Pathway Concentration Response Plot}{geneConcRespPlot}
%
\begin{Description}\relax
Plots a concentration response curve for one sample/signature combination.
\end{Description}
%
\begin{Usage}
\begin{verbatim}
geneConcRespPlot(row, plotrange = c(0.001, 100))
\end{verbatim}
\end{Usage}
%
\begin{Arguments}
\begin{ldescription}
\item[\code{row}] Named list containing:
\begin{itemize}

\item conc - conc string separated by |'s
\item resp - response string separated by |'s
\item method - scoring method determines plot bounds
\item proper\_name - chemical name for plot title
\item cutoff - noise cutoff
\item bmr - baseline median response; level at which bmd is calculated
\item er - fitted error term for plotting error bars
\item a, tp, b, ga, p, la, q - other model parameters for fit curve
\item fit\_method - curve fit method
\item bmd, bmdl, bmdu - bmd, bmd lower bound, and bmd upper bound
\item ac50, acc - curve value at 50
\item top - curve top
\item time, signature, signature\_class, signature\_size - other identifiers

\end{itemize}

Other elements are ignored.

\item[\code{plotrange}] The x-range of the plot as a vector of 2 elements, this can be changed for special cases, but defaults to 0.001 to 100
\end{ldescription}
\end{Arguments}
%
\begin{Details}\relax
row is one row of PATHWAY\_CR, the signatureConcResp output.
\end{Details}
%
\begin{Value}
No output.
\end{Value}
\inputencoding{utf8}
\HeaderA{geneConcRespPlotWrapper}{Wrapper for all of the conc-response plotting o genes}{geneConcRespPlotWrapper}
%
\begin{Description}\relax
Wrapper for all of the conc-response plotting o genes
\end{Description}
%
\begin{Usage}
\begin{verbatim}
geneConcRespPlotWrapper(
  dataset = "tox21_cpp5_heparg_pe1_normal",
  mc.cores = 20,
  do.load = T,
  to.file = F,
  pval = 0.05,
  plotrange = c(1e-04, 100),
  onefile = T,
  chemfile = NULL
)
\end{verbatim}
\end{Usage}
%
\begin{Arguments}
\begin{ldescription}
\item[\code{dataset}] Name of the data set.

\item[\code{mc.cores}] Number of cores to parallelize with.

\item[\code{do.load}] If TRUE, load the SIGNATURE\_CR file, otherwiseassume that it is in memory

\item[\code{to.file}] to.file = T saves the output to a file; otherwise it's returned.

\item[\code{pval}] Desired cutoff p-value.

\item[\code{plotrange}] The x-range of the plot as a vector of 2 elements, this can be changed for special cases, but defaults to 0.001 to 100

\item[\code{onefile}] If TRUE, put all plots into one file, instead of one filer per chemical

\item[\code{chemfile}] A file of chemicals to use. If NULL, plot all chemicals
\end{ldescription}
\end{Arguments}
\inputencoding{utf8}
\HeaderA{geneSlice}{Look at concentration-slides of gene CR data to understand where burst starts}{geneSlice}
%
\begin{Description}\relax
Look at concentration-slides of gene CR data to understand where burst starts
\end{Description}
%
\begin{Usage}
\begin{verbatim}
geneSlice(
  to.file = F,
  do.load = F,
  dataset = "mcf7_ph1_pe1_normal_block_123_allPG",
  celltype = "MCF7",
  cutoff = 0.9,
 
    chemfile = "../ERModel/ER_chems mcf7_ph1_pe1_normal_block_123_allPG estrogen 0.9 10.xlsx"
)
\end{verbatim}
\end{Usage}
%
\begin{Arguments}
\begin{ldescription}
\item[\code{to.file}] If TRUE, send the plots to a file

\item[\code{do.load}] If TRUE, load hte large HTTr data set into memory

\item[\code{dataset}] Name of the HTTr data set

\item[\code{celltype}] Name of the cell type

\item[\code{cutoff}] The minimum number of signatures hat have to be active in a super
target for the super target to be considered active. Default is 5

\item[\code{sigcatalog}] Name of the signature catalog to use

\item[\code{sigset}] Name of the signature set

\item[\code{method}] Scoring method

\item[\code{hccut}] Exclude rows in the data set with hitcall less than this value

\item[\code{tccut}] Exclude rows in the data set with top\_over\_cutoff less than this value

\item[\code{minconc}] Minimum concentration used in the plots

\item[\code{maxconc}] Maximum concentration used in the plots

After running this, run the following ...
superTargetPODplot
superTargetStats
\end{ldescription}
\end{Arguments}
\inputencoding{utf8}
\HeaderA{GSEA}{My Gene Set Enrichment Analysis}{GSEA}
%
\begin{Description}\relax
Performs tweaked version of single sample GSEA.
\end{Description}
%
\begin{Usage}
\begin{verbatim}
GSEA(
  X,
  geneSets,
  min.sz = 1,
  max.sz = Inf,
  alpha = 0.25,
  verbose = T,
  useranks = T
)
\end{verbatim}
\end{Usage}
%
\begin{Arguments}
\begin{ldescription}
\item[\code{X}] Transposed FCMAT2; i.e a gene by sample matrix of l2fc's including
rownames and colnames. Equivalent to expr in gsva.

\item[\code{geneSets}] Named list of signature definitions. Each element is a vector
of gene names. Each element name is a signature name.E quivalent to
gset.idx.list in gsva.

\item[\code{min.sz}] Minimum signature size (deprecated).

\item[\code{max.sz}] Maximum signature size (deprecated)

\item[\code{alpha}] Power of R to use. Higher alpha will upweight more extreme
ranks relative to middle ranks.

\item[\code{verbose}] verbose = T prints gene set length message.

\item[\code{useranks}] useranks = T uses ranks as in ssGSEA, while useranks = F
uses the bare fold changes instead.
\end{ldescription}
\end{Arguments}
%
\begin{Details}\relax
Based on the GSVA ssGSEA code. Main changes are: NAs are now handled correctly
and rank is now centered on zero instead of beginning at one. Since signature
sizes are undercounted here due to missing values, they are
assessed more accurately in signatureScoreCoreGSEA and limits are enforced
after scoring.
\end{Details}
%
\begin{Value}
Outputs signature by sample matrix of signature scores.
\end{Value}
%
\begin{Examples}
\begin{ExampleCode}
geneSets = list(signature1 = c("ABC", "DEF"), signature2 = c("ABC", "GHI"))
X = matrix(c(1:3,3:1), nrow = 3)
colnames(X) = c("Sample1", "Sample2")
rownames(X) = c("ABC", "DEF", "GHI")
GSEA(X,geneSets)
\end{ExampleCode}
\end{Examples}
\inputencoding{utf8}
\HeaderA{hello}{Hello, World!}{hello}
%
\begin{Description}\relax
Prints 'Hello, world!'.
\end{Description}
%
\begin{Usage}
\begin{verbatim}
hello()
\end{verbatim}
\end{Usage}
%
\begin{Examples}
\begin{ExampleCode}
hello()
\end{ExampleCode}
\end{Examples}
\inputencoding{utf8}
\HeaderA{mergePFASFCMAT1}{special code the merge the PFAS replacemnte data in with the earlier data fro U2OS and HepaRG}{mergePFASFCMAT1}
%
\begin{Description}\relax
version to start with Logan's database export
The difference between this version and the original is that there are extra columns
The function just changes one column name and writes the file to a standard name and place
\end{Description}
%
\begin{Usage}
\begin{verbatim}
mergePFASFCMAT1(
  dataset = "heparg2d_toxcast_pfas_pe1_normal_v2",
 
    file1 = "httr_heparg2d_toxcast_pfas_FCmat1-meanncnt0_5-plateteffect_1-shrinkage_normal.RData",
 
    file2 = "httr_pfas_replace_heparg_FCmat1-meanncnt0_5-plateteffect_1-shrinkage_normal.RData",
  dir = "../input/fcdata/new_versions/",
  do.load = T
)
\end{verbatim}
\end{Usage}
%
\begin{Arguments}
\begin{ldescription}
\item[\code{dataset}] The name to give to the data set

\item[\code{dir}] The directory from which to read all of the raw files

\item[\code{do.load}] If TRUE, read the large input data file into memory

\item[\code{infile}] The nae of the input file

\item[\code{pg.filter.file}] An optional file to use in filtering out bad plate groups
\end{ldescription}
\end{Arguments}
%
\begin{Value}
A file with the FCMAT1 data is written to "../input/fcdata/FCMAT1\_",dataset,".RData"

* heparg2d\_toxcast\_pfas\_pe1\_normal
* u2os\_toxcast\_pfas\_pe1\_normal
\end{Value}
\inputencoding{utf8}
\HeaderA{pg\_id.to.sample\_id}{get the mapping between the plate groups and the samples}{pg.Rul.id.to.sample.Rul.id}
%
\begin{Description}\relax
version to start with Logan's database export
The difference between this version and the original is that there are extra columns
The function just changes one column name and writes the file to a standard name and place
\end{Description}
%
\begin{Usage}
\begin{verbatim}
pg_id.to.sample_id(
  do.load = F,
  dataset = "mcf7_ph1_pe1_normal_block_123_allPG",
  dir = "../input/fcdata/"
)
\end{verbatim}
\end{Usage}
%
\begin{Arguments}
\begin{ldescription}
\item[\code{do.load}] if T, load the initial file

\item[\code{dataset}] The name to give to the data set

\item[\code{dir}] The directory from which to read all of the raw files
\end{ldescription}
\end{Arguments}
%
\begin{Value}
A mapping the sampels to the plate groups
f
\end{Value}
\inputencoding{utf8}
\HeaderA{plotouter}{Plot Outer}{plotouter}
%
\begin{Description}\relax
Calls signatureConcResp plotting function.
\end{Description}
%
\begin{Usage}
\begin{verbatim}
plotouter(proper_name, SIGNATURE_CR, foldname, plotrange = c(0.001, 100))
\end{verbatim}
\end{Usage}
%
\begin{Arguments}
\begin{ldescription}
\item[\code{proper\_name}] Chemical name to be used in file name.

\item[\code{SIGNATURE\_CR}] Dataframe output of signatureConcResp\_pval.

\item[\code{foldname}] Folder name for output file.

\item[\code{plotrange}] The x-range of the plot as a vector of 2 elements, this can be changed for special cases, but defaults to 0.001 to 100
\end{ldescription}
\end{Arguments}
%
\begin{Details}\relax
Calls signatureConcResp plotting function for one chemical and every signature.
Saves a single pdf to disk for the given chemical containing every signature
CR plot.
\end{Details}
%
\begin{Value}
No output.
\end{Value}
\inputencoding{utf8}
\HeaderA{plotouterGene}{Plot Outer}{plotouterGene}
%
\begin{Description}\relax
Calls signatureConcResp plotting function.
\end{Description}
%
\begin{Usage}
\begin{verbatim}
plotouterGene(proper_name, GENE_CR, foldname, plotrange = c(0.001, 100))
\end{verbatim}
\end{Usage}
%
\begin{Arguments}
\begin{ldescription}
\item[\code{proper\_name}] Chemical name to be used in file name.

\item[\code{GENE\_CR}] Dataframe output of geneConcResp\_pval.

\item[\code{foldname}] Folder name for output file.

\item[\code{plotrange}] The x-range of the plot as a vector of 2 elements, this can be changed for special cases, but defaults to 0.001 to 100
\end{ldescription}
\end{Arguments}
%
\begin{Details}\relax
Calls signatureConcResp plotting function for one chemical and every signature.
Saves a single pdf to disk for the given chemical containing every signature
CR plot.
\end{Details}
%
\begin{Value}
No output.
\end{Value}
\inputencoding{utf8}
\HeaderA{podLaneplot}{Build lane plots by chemical list and signature class, across the datasets}{podLaneplot}
%
\begin{Description}\relax
Build lane plots by chemical list and signature class, across the datasets
\end{Description}
%
\begin{Usage}
\begin{verbatim}
podLaneplot(
  to.file = F,
  dataset = "DMEM_6hr_pilot_normal_pe_1",
  sigset = "pilot_large_all_100CMAP",
  method = "gsea",
  hccut = 0.9,
  plot.signature_min = F,
  bmd.mode = "percent"
)
\end{verbatim}
\end{Usage}
%
\begin{Arguments}
\begin{ldescription}
\item[\code{to.file}] If TRUE, write plots to a file

\item[\code{dataset}] The data set to use

\item[\code{sigset}] THe signature set to use

\item[\code{method}] Scoring method

\item[\code{hccut}] Exclude rows with hitcall less than this value

\item[\code{bmd.mode}] percent or abs

\item[\code{plot.signature.min}] If TRUE, plot the minimum signature
\end{ldescription}
\end{Arguments}
\inputencoding{utf8}
\HeaderA{printCurrentFunction}{Print the name of the current function}{printCurrentFunction}
%
\begin{Description}\relax
Print the name of the current function
\end{Description}
%
\begin{Usage}
\begin{verbatim}
printCurrentFunction(comment.string = NA)
\end{verbatim}
\end{Usage}
%
\begin{Arguments}
\begin{ldescription}
\item[\code{comment.string}] An optinal string to be printed
\end{ldescription}
\end{Arguments}
\inputencoding{utf8}
\HeaderA{R2}{R Squared}{R2}
%
\begin{Description}\relax
Calculate coefficient of determination.
\end{Description}
%
\begin{Usage}
\begin{verbatim}
R2(y, pred)
\end{verbatim}
\end{Usage}
%
\begin{Arguments}
\begin{ldescription}
\item[\code{y}] Vector of actual values.

\item[\code{pred}] Vector of corresponding predicted values.
\end{ldescription}
\end{Arguments}
%
\begin{Details}\relax
Note that order matters: R2(x,y) != R2(y,x) in general.
\end{Details}
%
\begin{Value}
Coefficient of determination.
\end{Value}
%
\begin{Examples}
\begin{ExampleCode}
R2(c(1:10), c(1:10*.8))
R2(c(1:10*.8), c(1:10))
\end{ExampleCode}
\end{Examples}
\inputencoding{utf8}
\HeaderA{randomdata}{Randomized Null Data}{randomdata}
%
\begin{Description}\relax
Generate randomized null data based on actual data.
\end{Description}
%
\begin{Usage}
\begin{verbatim}
randomdata(
  basedir = "../input/fcdata/",
  dataset = "u2os_pilot_pe1_normal_null_pilot_lowconc",
  nchem = 1000,
  seed = 12345,
  maxconc = 1e+06,
  nlowconc = 2,
  dtxsid.exclude = NULL
)
\end{verbatim}
\end{Usage}
%
\begin{Arguments}
\begin{ldescription}
\item[\code{basedir}] Directory that holds FCMAT2 and CHEM\_DICT files.

\item[\code{dataset}] Name of actual dataset to base null data on.

\item[\code{nchem}] Number of null chemicals. Number of null samples is approximately
eight times this value.

\item[\code{seed}] Random seed.

\item[\code{maxconc}] Only use concentrations less than maxconc, default 1000000

\item[\code{nlowconc}] If not NULL, only include the loest nlowconc concentrations for eahc chemical

\item[\code{dtxsid.exclude}] dtxsids to exclude, default NULL
for U2OS pilot dtxsid.exclude=c('DTXSID9020031','DTXSID0040464','DTXSID5023582')
\end{ldescription}
\end{Arguments}
%
\begin{Details}\relax
New FCMAT2 and CHEM\_DICT files corresponding to the null dataset are written
to disk in the basedir folder. The nullset name is paste0(dataset, "\_", nchem).
Randomization is performed by sampling the quantile function for each gene in
the actual data. The nullset will have roughly the same distribution of values
for each gene in the actual data,
\end{Details}
%
\begin{Value}
No output.
\end{Value}
\inputencoding{utf8}
\HeaderA{RMSE}{Root-mean-square-error}{RMSE}
%
\begin{Description}\relax
Computes root-mean-square-error between two vectors.
\end{Description}
%
\begin{Usage}
\begin{verbatim}
RMSE(x, y)
\end{verbatim}
\end{Usage}
%
\begin{Arguments}
\begin{ldescription}
\item[\code{x}] First vector.

\item[\code{y}] Second vector.
\end{ldescription}
\end{Arguments}
%
\begin{Value}
RMSE
\end{Value}
%
\begin{Examples}
\begin{ExampleCode}
RMSE(1:3, c(1,3,5))
\end{ExampleCode}
\end{Examples}
\inputencoding{utf8}
\HeaderA{runAllSignatureCR}{Run All Pathway Concentration Response (P-Value)}{runAllSignatureCR}
%
\begin{Description}\relax
Driver for signature scoring and concentration response (CR).
\end{Description}
%
\begin{Usage}
\begin{verbatim}
runAllSignatureCR(
  dataset,
  sigset,
  cutoff.dataset,
  sigcatalog,
  method,
  bmr_scale = 1.349,
  normfactor = 7500,
  minsigsize = 10,
  pval = 0.05,
  nlowconc = 2,
  mc.cores = 1,
  fitmodels = c("cnst", "hill", "poly1", "poly2", "pow", "exp2", "exp3", "exp4",
    "exp5")
)
\end{verbatim}
\end{Usage}
%
\begin{Arguments}
\begin{ldescription}
\item[\code{dataset}] Name of data set.

\item[\code{sigset}] Name of signature set.

\item[\code{cutoff.dataset}] This is the data set name to sue when the cutoffs are taken from a different data set than
the one currently being analyzed. The reason for doing this is if the current data set is small
(small number of chemicals), and so not large enough to get a good estiamte of the underlying
noise distribution. All of the other parameters for both data sets have to be the same

\item[\code{sigcatalog}] Name of the signature catalog

\item[\code{method}] Pathway scoring method in c("fc", "gsva", "gsea")

\item[\code{bmr\_scale}] bmr scaling factor. Default = 1.349

\item[\code{normfactor}] Factor to scale the native units up by to get onto a reasonable plotting value (\textasciitilde{} -1 to 1)

\item[\code{minsigsize}] Minimum signature size.

\item[\code{pval}] P-value to use for noise estimation.

\item[\code{nlowconc}] Only include the lowest nlowconc concentrations for each chemical

\item[\code{mc.cores}] Vector with two values: number of cores to use for signature
scoring and number of cores to use for CR. CR can usually handle the maximum
number, but gsva scoring might require a smaller number to avoid memory
overflow.

\item[\code{fitmodels}] Vector of model names to run conc/resp with. "cnst" should
always be chosen.
\end{ldescription}
\end{Arguments}
%
\begin{Details}\relax
Signature scores are written to disk in output/signature\_score\_summary/.
Signature cutoffs are written to disk in output/signature\_cutoff/.
CR results are written to disk in output/signature\_conc\_resp\_summary/.
\end{Details}
%
\begin{Value}
No output.

remove gnls from default set
\end{Value}
\inputencoding{utf8}
\HeaderA{signatureCatalogLoader}{Merge the up and down halves of the pathway data}{signatureCatalogLoader}
%
\begin{Description}\relax
Merge the up and down halves of the pathway data
\end{Description}
%
\begin{Usage}
\begin{verbatim}
signatureCatalogLoader(
  sigset = "wgcna",
  sigcatalog = "signatureDB_wgcna_mcf7_ph1_pe1_normal_good_pg_MCF7_12_10_catalog"
)
\end{verbatim}
\end{Usage}
%
\begin{Arguments}
\begin{ldescription}
\item[\code{sigset}] Name of the signature set.

\item[\code{sigcatlog}] Nmae of the catalog file
\end{ldescription}
\end{Arguments}
%
\begin{Value}
the trimmed signature table
\end{Value}
\inputencoding{utf8}
\HeaderA{signatureConcRepFilter}{Filter the conc-repons data}{signatureConcRepFilter}
%
\begin{Description}\relax
Filter the conc-repons data
\end{Description}
%
\begin{Usage}
\begin{verbatim}
signatureConcRepFilter(
  to.file = F,
  do.plot = F,
  do.load = T,
  hccut = 0.9,
  tccut = 1.5,
  dataset = "heparg2d_toxcast_pfas_pe1_normal",
  sigset = "screen_large",
  method = "fc",
  do.pfas = F
)
\end{verbatim}
\end{Usage}
%
\begin{Arguments}
\begin{ldescription}
\item[\code{method}] signature scoring method in c("fc", "gsva", "mygsea")


Error bars are exp(er)*qt(.025,4) = exp(er)*2.7765
heparg2d\_toxcast\_pfas\_pe1\_normal
mcf7\_ph1\_pe1\_normal\_block\_123
mcf7\_ph1\_pe1\_normal\_block\_123\_allPG
u2os\_toxcast\_pfas\_pe1\_normal
\end{ldescription}
\end{Arguments}
\inputencoding{utf8}
\HeaderA{signatureConcResp}{Pathway Concentration Response (P-value)}{signatureConcResp}
%
\begin{Description}\relax
Performs signature concentration response using p-value based cutoffs.
\end{Description}
%
\begin{Usage}
\begin{verbatim}
signatureConcResp(
  dataset,
  sigset,
  cutoff.dataset,
  sigcatalog,
  method,
  bmr_scale = 1.349,
  mc.cores = 1,
  pval = 0.05,
  nlowconc = 2,
  aicc = F,
  minsigsize = 10,
  fitmodels = c("cnst", "hill", "gnls", "poly1", "poly2", "pow", "exp2", "exp3",
    "exp4", "exp5")
)
\end{verbatim}
\end{Usage}
%
\begin{Arguments}
\begin{ldescription}
\item[\code{dataset}] Name of the data set.

\item[\code{sigset}] Name of the signature set.

\item[\code{cutoff.dataset}] This is the data set name to sue when the cutoffs are taken from a different data set than
the one currently being analyzed. The reason for doing this is if the current data set is small
(small number of chemicals), and so not large enough to get a good estiamte of the underlying
noise distribution. All of the other parameters for both data sets have to be the same

\item[\code{method}] Pathway scoring method in c("fc", "gsva", "gsea")

\item[\code{bmr\_scale}] bmr scaling factor. Default = 1.349

\item[\code{mc.cores}] Number of cores to parallelize with.

\item[\code{pval}] Desired cutoff p-value.

\item[\code{nlowconc}] Only include the lowest nlowconc concentrations for each chemical

\item[\code{aicc}] aicc = T uses corrected AIC to choose winning method; otherwise
regular AIC.

\item[\code{minsigsize}] Minimum allowed signature size. Sample/signature combinations
with less than this number of non-missing l2fc's will be discarded.

\item[\code{fitmodels}] Vector of model names to use. Probably should include "cnst".
\end{ldescription}
\end{Arguments}
%
\begin{Details}\relax
dataset should have already been scored using signatureScore
and the given sigset and method. This function prepares signatureScore output
for CR processing, calls signatureConcRespCore\_pval, formats the output,
saves it to disk
\end{Details}
%
\begin{Value}
If to.file = T, nothing. If to.file = F, dataframe with signature CR
output.
\end{Value}
\inputencoding{utf8}
\HeaderA{signatureConcRespFilter}{Filter the conc-response data for just the most potent results and plot the conc-response curves if desired}{signatureConcRespFilter}
%
\begin{Description}\relax
Filter the conc-response data for just the most potent results
and plot the conc-response curves if desired
\end{Description}
%
\begin{Usage}
\begin{verbatim}
signatureConcRespFilter(
  to.file = F,
  do.plot = F,
  do.load = T,
  hccut = 0.9,
  tccut = 1.5,
  dataset = "heparg2d_toxcast_pfas_pe1_normal",
  sigset = "screen_large",
  method = "fc",
  do.pfas = F
)
\end{verbatim}
\end{Usage}
%
\begin{Arguments}
\begin{ldescription}
\item[\code{to.file}] If TRUE, send plots to a file

\item[\code{do.plot}] If TRUE do the plotting

\item[\code{do.load}] If TRUE, load the data file

\item[\code{hccut}] Exclude rows with hitcall below this value

\item[\code{tccut}] Exclude rows with top\_over\_cutoff below this value

\item[\code{dataset}] Dataset to use

\item[\code{sigset}] Signature set to use

\item[\code{method}] signature scoring method in c("fc", "gsva", "gsea")

\item[\code{do.pfas=F}] Error bars are exp(er)*qt(.025,4) = exp(er)*2.7765
\end{ldescription}
\end{Arguments}
\inputencoding{utf8}
\HeaderA{signatureConcRespPlot}{Pathway Concentration Response Plot}{signatureConcRespPlot}
%
\begin{Description}\relax
Plots a concentration response curve for one sample/signature combination.
\end{Description}
%
\begin{Usage}
\begin{verbatim}
signatureConcRespPlot(row, plotrange = c(0.001, 100))
\end{verbatim}
\end{Usage}
%
\begin{Arguments}
\begin{ldescription}
\item[\code{row}] Named list containing:
\begin{itemize}

\item conc - conc string separated by |'s
\item resp - response string separated by |'s
\item method - scoring method determines plot bounds
\item proper\_name - chemical name for plot title
\item cutoff - noise cutoff
\item bmr - baseline median response; level at which bmd is calculated
\item er - fitted error term for plotting error bars
\item a, tp, b, ga, p, la, q - other model parameters for fit curve
\item fit\_method - curve fit method
\item bmd, bmdl, bmdu - bmd, bmd lower bound, and bmd upper bound
\item ac50, acc - curve value at 50
\item top - curve top
\item time, signature, signature\_class, signature\_size - other identifiers

\end{itemize}

Other elements are ignored.

\item[\code{plotrange}] The x-range of the plot as a vector of 2 elements, this can be changed for special cases, but defaults to 0.001 to 100
\end{ldescription}
\end{Arguments}
%
\begin{Details}\relax
row is one row of PATHWAY\_CR, the signatureConcResp output.
\end{Details}
%
\begin{Value}
No output.
\end{Value}
\inputencoding{utf8}
\HeaderA{signatureConcRespPlotWrapper}{Wrapper for all of the conc-response plotting}{signatureConcRespPlotWrapper}
%
\begin{Description}\relax
Wrapper for all of the conc-response plotting
\end{Description}
%
\begin{Usage}
\begin{verbatim}
signatureConcRespPlotWrapper(
  sigset,
  dataset,
  sigcatalog,
  method,
  bmr_scale = 1.349,
  mc.cores = 20,
  do.load = T,
  pval = 0.05,
  plotrange = c(1e-04, 100)
)
\end{verbatim}
\end{Usage}
%
\begin{Arguments}
\begin{ldescription}
\item[\code{sigset}] Name of the signature set.

\item[\code{dataset}] Name of the data set.

\item[\code{sigcatalog}] Name of the signature catalog

\item[\code{method}] Pathway scoring method in c("fc", "gsva", "gsea")

\item[\code{bmr\_scale}] bmr scaling factor. Default = 1.349

\item[\code{mc.cores}] Number of cores to parallelize with.

\item[\code{do.load}] If TRUE, load the SIGNATURE\_CR file, otherwiseassume that it is in memory
to.file to.file = T saves the output to a file; otherwise it's returned.

\item[\code{pval}] Desired cutoff p-value.

\item[\code{plotrange}] The x-range of the plot as a vector of 2 elements, this can be changed for special cases, but defaults to 0.001 to 100
\end{ldescription}
\end{Arguments}
\inputencoding{utf8}
\HeaderA{signatureConcRespToZ}{Convert the conc-response data to a z score}{signatureConcRespToZ}
%
\begin{Description}\relax
Convert the conc-response data to a z score
\end{Description}
%
\begin{Usage}
\begin{verbatim}
signatureConcRespToZ(
  do.load = T,
  mc.cores = 2,
  dataset = "heparg2d_toxcast_pfas_pe1_normal",
  sigset = "screen_large",
  method = "fc",
  celltype = "HepaRG",
  hccut = 0.95,
  tccut = 1.5
)
\end{verbatim}
\end{Usage}
%
\begin{Arguments}
\begin{ldescription}
\item[\code{do.load}] If TRUE, load hte large HTTr data set

\item[\code{mc.cores}] NUmber of cores to use in multi-core mode=2,

\item[\code{dataset}] Name of the HTTr data set being used

\item[\code{sigset}] Name of the signature set used

\item[\code{method}] Scoring method used

\item[\code{celltype}] name of cell type ebing used

\item[\code{hccut}] Exclude signature rows with hitcall less than this value

\item[\code{tccut}] Exclude signature rows with top\_over\_cutoff less than this value
\end{ldescription}
\end{Arguments}
\inputencoding{utf8}
\HeaderA{signatureDirectionPlot}{Plot the cummulative distribution functions of the up and down direction signatures}{signatureDirectionPlot}
%
\begin{Description}\relax
Plot the cummulative distribution functions of the up and down direction signatures
\end{Description}
%
\begin{Usage}
\begin{verbatim}
signatureDirectionPlot(
  to.file = T,
  do.load = F,
  dataset = "MCF7_pilot_PRF_6hr_pilot_normal_pe_1",
  sigset = "screen_large",
  method = "gsea",
  celltype = "MCF7",
  hccut = 0.9,
  tccut = 1
)
\end{verbatim}
\end{Usage}
%
\begin{Arguments}
\begin{ldescription}
\item[\code{to.file}] If TRUE, send the plots to a file

\item[\code{do.load}] If TRUE, load hte large HTTr data set into memory

\item[\code{dataset}] Name of the HTTr data set

\item[\code{sigset}] Name of the signature set

\item[\code{method}] Scoring method

\item[\code{celltype}] Name of the cell type

\item[\code{hccut}] Exclude rows in the data set with hitcall less than this value

\item[\code{tccut}] Exclude rows in the data set with top\_over\_cutoff less than this value

\item[\code{sigcatalog}] Name of the signature catalog to use

\item[\code{cutoff}] The minimum number of signatures hat have to be active in a super
target for the super target to be considered active. Default is 5

\item[\code{minconc}] Minimum concentration used in the plots

\item[\code{maxconc}] Maximum concentration used in the plots

After running this, run the following ...
superTargetPODplot
superTargetStats
\end{ldescription}
\end{Arguments}
\inputencoding{utf8}
\HeaderA{signatureFinder}{Find signatures out o a set of chemicals}{signatureFinder}
%
\begin{Description}\relax
Find signatures out o a set of chemicals
\end{Description}
%
\begin{Usage}
\begin{verbatim}
signatureFinder(
  to.file = F,
  do.load = F,
  dataset = "mcf7_ph1_pe1_normal_block_123_allPG",
  celltype = "MCF7",
  ngene = 200,
  cutoff = 0.9,
 
    chemfile = "../ERModel/ER_chems all mcf7_ph1_pe1_normal_block_123_allPG screen_large 0.9 10.xlsx"
)
\end{verbatim}
\end{Usage}
%
\begin{Arguments}
\begin{ldescription}
\item[\code{to.file}] If TRUE, send the plots to a file

\item[\code{do.load}] If TRUE, load hte large HTTr data set into memory

\item[\code{dataset}] Name of the HTTr data set

\item[\code{celltype}] Name of the cell type

\item[\code{cutoff}] The minimum number of signatures hat have to be active in a super
target for the super target to be considered active. Default is 5

\item[\code{sigcatalog}] Name of the signature catalog to use

\item[\code{sigset}] Name of the signature set

\item[\code{method}] Scoring method

\item[\code{hccut}] Exclude rows in the data set with hitcall less than this value

\item[\code{tccut}] Exclude rows in the data set with top\_over\_cutoff less than this value

\item[\code{minconc}] Minimum concentration used in the plots

\item[\code{maxconc}] Maximum concentration used in the plots

After running this, run the following ...
superTargetPODplot
superTargetStats
\end{ldescription}
\end{Arguments}
\inputencoding{utf8}
\HeaderA{signaturePOD}{Calculate PODs at the signature level}{signaturePOD}
%
\begin{Description}\relax
Calculate PODs at the signature level
\end{Description}
%
\begin{Usage}
\begin{verbatim}
signaturePOD(
  do.load = F,
  sigset = "screen_large",
  dataset = "MCF7_pilot_DMEM_6hr_pilot_normal_pe_1",
  method = "gsea",
  hccut = 0.9,
  cutoff = 3,
  condition = "all"
)
\end{verbatim}
\end{Usage}
%
\begin{Arguments}
\begin{ldescription}
\item[\code{sigset}] Name of signature set.

\item[\code{dataset}] Name of data set.

\item[\code{method}] Pathway scoring method in c("fc", "gsva", "gsea")

\item[\code{hccut}] Remove rows with hitcall less than this value

\item[\code{do.laod}] If TRUE, load the input data into memory

\item[\code{bmr\_scale}] bmr scaling factor. Default = 1.349
\end{ldescription}
\end{Arguments}
\inputencoding{utf8}
\HeaderA{signaturePOD.BMRcompare}{Compare the PODs with different BMR values}{signaturePOD.BMRcompare}
%
\begin{Description}\relax
Compare the PODs with different BMR values
\end{Description}
%
\begin{Usage}
\begin{verbatim}
signaturePOD.BMRcompare(
  to.file = F,
  dataset = "mcf7_ph1_pe1_normal_block_123",
  sigset = "screen_large",
  method = "fc",
  bmr_scale = 1,
  hccut = 0.9
)
\end{verbatim}
\end{Usage}
%
\begin{Arguments}
\begin{ldescription}
\item[\code{to.file}] If TRUE, write plots to a file

\item[\code{dataset}] Name of data set.

\item[\code{sigset}] Name of signature set.

\item[\code{method}] Pathway scoring method

\item[\code{bmr\_scale}] bmr scaling factor. Default = 1.349

\item[\code{hccut}] Remove rows with hitcall less than this value
\end{ldescription}
\end{Arguments}
\inputencoding{utf8}
\HeaderA{signaturePODsummary}{Summarize the POD overlap with ToxCast}{signaturePODsummary}
%
\begin{Description}\relax
Summarize the POD overlap with ToxCast
\end{Description}
%
\begin{Usage}
\begin{verbatim}
signaturePODsummary(
  sigset = "pilot_large_all_100CMAP",
  dataset = "DMEM_6hr_pilot_normal_pe_1",
  method = "gsea"
)
\end{verbatim}
\end{Usage}
%
\begin{Arguments}
\begin{ldescription}
\item[\code{sigset}] THe name of the signature set

\item[\code{dataset}] Name of the HTTr data set

\item[\code{method}] THe signature scoring method
\end{ldescription}
\end{Arguments}
\inputencoding{utf8}
\HeaderA{signatureScore}{Signature Score}{signatureScore}
%
\begin{Description}\relax
Computes and saves signature scores.
\end{Description}
%
\begin{Usage}
\begin{verbatim}
signatureScore(
  FCMAT2,
  CHEM_DICT,
  sigset,
  sigcatalog,
  dataset,
  method,
  normfactor = 7500,
  mc.cores = 1,
  minsigsize = 10
)
\end{verbatim}
\end{Usage}
%
\begin{Arguments}
\begin{ldescription}
\item[\code{FCMAT2}] Sample by gene matrix of log2(fold change)'s. Rownames are
sample keys and colnames are genes.

\item[\code{CHEM\_DICT}] Dataframe with one row per sample key and seven columns:
sample\_key, sample\_id, conc, time, casrn, name, dtxsid.

\item[\code{sigset}] Name of signature set.

\item[\code{sigcatalog}] Name of the signature catalog file

\item[\code{dataset}] Name of data set.

\item[\code{method}] Signature scoring method in c("fc", "gsva", "gsea")

\item[\code{normfactor}] Value passed ot the plotting code to scale the y values

\item[\code{mc.cores}] Number of cores to use.

\item[\code{minsigsize}] Minimum allowed signature size BEFORE accounting for
missing values.
\end{ldescription}
\end{Arguments}
%
\begin{Details}\relax
signatureScore is a driver for various scoring methods. The three that are
currently available are "gsva", "gsea", "fc", and "gsea\_norank" (a version
of gsea that uses fold changes instead of ranks as weights). Deprecated
methods include the Fisher method and gsvae (gsva with empirical cdfs).
Beware running out of memory on large runs with gsva, Linux, and many cores.
Signature size is counted according to number of genes in the signature that are
also in the column names of FCMAT2. However, each method performs a more
rigorous size count internally that accounts for missing values and adds this
to the output. This minsigsize is enforced when running signatureConcResp\_pval.
\end{Details}
%
\begin{Value}
No output.
\end{Value}
\inputencoding{utf8}
\HeaderA{signatureScoreCoreFC}{Signature Score Core - FC}{signatureScoreCoreFC}
%
\begin{Description}\relax
Computes fold change signature scores.
\end{Description}
%
\begin{Usage}
\begin{verbatim}
signatureScoreCoreFC(
  fcdata,
  sigset,
  dataset,
  chem_dict,
  signature_data,
  ngenemax = NULL,
  verbose = F
)
\end{verbatim}
\end{Usage}
%
\begin{Arguments}
\begin{ldescription}
\item[\code{fcdata}] Sample by gene matrix of log2(fold change)'s. Rownames are
sample keys and colnames are genes.

\item[\code{sigset}] Name of signature set.

\item[\code{dataset}] Name of data set.

\item[\code{chem\_dict}] Dataframe with one row per sample key and seven columns:
sample\_key, sample\_id, conc, time, casrn, name, dtxsid.

\item[\code{signature\_data}] Named ist of gene name vectors. Each element is one
signature, defined by the genes it contains.

\item[\code{ngenemax}] If ngene is not NULL, then tonly the most extreme n genes of the
signature will be used for the "in" set

\item[\code{verbose}] If TRUE, weite extra diagnostic output
\end{ldescription}
\end{Arguments}
%
\begin{Details}\relax
This fast implementation of fold change signature scores uses matrix
multiplication. The score is simply: mean(fold change of genes in signature) -
mean(fold change of genes outside signature).
\end{Details}
%
\begin{Value}
Dataframe with one row per chemical/conc/signature combination. Columns
are: sample\_id, dtxsid, casrn, name, time, conc, sigset,
signature, size (signature size accounting for missing values), mean\_fc\_scaled\_in,
mean\_fc\_scaled\_out, signature\_score.
\end{Value}
\inputencoding{utf8}
\HeaderA{signatureScoreCoreGSEA}{Signature Score Core - GSEA}{signatureScoreCoreGSEA}
%
\begin{Description}\relax
Computes signature scores for gsea.
\end{Description}
%
\begin{Usage}
\begin{verbatim}
signatureScoreCoreGSEA(
  sk.list,
  method = "gsea",
  normfactor = 7500,
  sigset,
  dataset,
  fcmat,
  chem_dict,
  signature_data,
  mc.cores = 1,
  normalization = T,
  useranks = T
)
\end{verbatim}
\end{Usage}
%
\begin{Arguments}
\begin{ldescription}
\item[\code{sk.list}] Sample keys to use; should correspond to fcmat rownames.

\item[\code{method}] Method name to use in file output. "gsea" or "gsea\_norank"

\item[\code{sigset}] Name of signature set.

\item[\code{dataset}] Name of data set.

\item[\code{fcmat}] Sample by gene matrix of log2(fold change)'s. Rownames are
sample keys and colnames are genes.

\item[\code{chem\_dict}] Dataframe with one row per sample key and seven columns:
sample\_key, sample\_id, conc, time, casrn, name, dtxsid.

\item[\code{signature\_data}] Named ist of gene name vectors. Each element is one
signature, defined by the genes it contains.

\item[\code{mc.cores}] Number of cores to use. Parallelization is performed
by gsva itself.

\item[\code{normalization}] normalization = T normalizes final scores.

\item[\code{useranks}] useranks = T uses score ranks for weighting; otherwise,
fold changes are used for weights.
\end{ldescription}
\end{Arguments}
%
\begin{Details}\relax
This function is a parallelized wrapper for gsea, which does the actual
scoring. gsea method uses ranks and normalization, while gsea\_norank
method does not use ranks or normalization. Normalization divides final
scores by difference between max and min score. Without normalization,
scores from individual samples have no impact on each other. Final
signaturescoremat is written to disk.
\end{Details}
%
\begin{Value}
No output.
\end{Value}
\inputencoding{utf8}
\HeaderA{signatureScoreCoreGSVA}{Signature Score Core - GSVA}{signatureScoreCoreGSVA}
%
\begin{Description}\relax
Computes GSVA signature scores.
\end{Description}
%
\begin{Usage}
\begin{verbatim}
signatureScoreCoreGSVA(
  sk.list,
  sigset = "FILTERED",
  dataset,
  fcmat,
  chem_dict,
  signature_data,
  mc.cores = 1
)
\end{verbatim}
\end{Usage}
%
\begin{Arguments}
\begin{ldescription}
\item[\code{sk.list}] Sample keys to use; should correspond to fcmat rownames.

\item[\code{sigset}] Name of signature set.

\item[\code{dataset}] Name of data set.

\item[\code{fcmat}] Sample by gene matrix of log2(fold change)'s. Rownames are
sample keys and colnames are genes.

\item[\code{chem\_dict}] Dataframe with one row per sample key and seven columns:
sample\_key, sample\_id, conc, time, casrn, name, dtxsid.

\item[\code{signature\_data}] Named ist of gene name vectors. Each element is one
signature, defined by the genes it contains.

\item[\code{mc.cores}] Number of cores to use. Parallelization is performed
by gsva itself.
\end{ldescription}
\end{Arguments}
%
\begin{Details}\relax
This function is a wrapper for GSVA with Gaussian cdf kernels. signaturescoremat
output is saved directly to disk.
\end{Details}
%
\begin{Value}
No output.
\end{Value}
\inputencoding{utf8}
\HeaderA{signatureScoreMerge}{Merge the up and down halves of the pathway data}{signatureScoreMerge}
%
\begin{Description}\relax
Merge the up and down halves of the pathway data
\end{Description}
%
\begin{Usage}
\begin{verbatim}
signatureScoreMerge(sigset, sigcatalog, dataset, method)
\end{verbatim}
\end{Usage}
%
\begin{Arguments}
\begin{ldescription}
\item[\code{sigset}] Name of the signature set.

\item[\code{dataset}] Name of the data set.

\item[\code{method}] Pathway scoring method in c("fc", "gsva", "gsea")

\item[\code{sigcatlog}] Name of the catalog file
\end{ldescription}
\end{Arguments}
%
\begin{Value}
nothing
\end{Value}
\inputencoding{utf8}
\HeaderA{signatureSlice}{Look at concentration-slides of gene CR data to understand where burst starts}{signatureSlice}
%
\begin{Description}\relax
Look at concentration-slides of gene CR data to understand where burst starts
\end{Description}
%
\begin{Usage}
\begin{verbatim}
signatureSlice(
  to.file = F,
  do.load = F,
  dataset = "mcf7_ph1_pe1_normal_block_123_allPG",
  sigcatalog = "signatureDB_master_catalog 2021-09-29",
  sigset = "screen_large",
  method = "gsea",
  celltype = "MCF7",
  hccut = 0.9,
  minhit = 10,
  tccut = 0.9
)
\end{verbatim}
\end{Usage}
%
\begin{Arguments}
\begin{ldescription}
\item[\code{to.file}] If TRUE, send the plots to a file

\item[\code{do.load}] If TRUE, load hte large HTTr data set into memory

\item[\code{dataset}] Name of the HTTr data set

\item[\code{sigcatalog}] Name of the signature catalog to use

\item[\code{sigset}] Name of the signature set

\item[\code{method}] Scoring method

\item[\code{celltype}] Name of the cell type

\item[\code{hccut}] Exclude rows in the data set with hitcall less than this value

\item[\code{tccut}] Exclude rows in the data set with top\_over\_cutoff less than this value

\item[\code{cutoff}] The minimum number of signatures hat have to be active in a super
target for the super target to be considered active. Default is 5

\item[\code{minconc}] Minimum concentration used in the plots

\item[\code{maxconc}] Maximum concentration used in the plots

After running this, run the following ...
superTargetPODplot
superTargetStats
\end{ldescription}
\end{Arguments}
\inputencoding{utf8}
\HeaderA{superTargetBoxplot}{Generate chemical-wise boxplot of the BMD distributions by super\_target}{superTargetBoxplot}
%
\begin{Description}\relax
Generate chemical-wise boxplot of the BMD distributions by super\_target
\end{Description}
%
\begin{Usage}
\begin{verbatim}
superTargetBoxplot(
  to.file = T,
  do.load = T,
  dataset = "u2os_toxcast_pfas_pe1_normal_v2_refchems",
  sigcatalog = "signatureDB_master_catalog 2021-09-29",
  sigset = "screen_large",
  method = "gsea",
  celltype = "U2OS",
  hccut = 0.9,
  tccut = 1,
  cutoff = 3,
  minconc = 0.001,
  maxconc = 100,
  chemfile = NULL
)
\end{verbatim}
\end{Usage}
%
\begin{Arguments}
\begin{ldescription}
\item[\code{to.file}] If TRUE, send the plots to a file

\item[\code{do.load}] If TRUE, load hte large HTTr data set into memory

\item[\code{dataset}] Name of the HTTr data set

\item[\code{sigcatalog}] Name of the signature catalog to use

\item[\code{sigset}] Name of the signature set

\item[\code{method}] Scoring method

\item[\code{celltype}] Name of the cell type

\item[\code{hccut}] Exclude rows in the data set with hitcall less than this value

\item[\code{tccut}] Exclude rows in the data set with top\_over\_cutoff less than this value

\item[\code{cutoff}] The minimum number of signatures hat have to be active in a super
target for the super target to be considered active. Default is 5

\item[\code{minconc}] Minimum concentration used in the plots

\item[\code{maxconc}] Maximum concentration used in the plots

After running this, run the following ...
superTargetPODplot
superTargetStats
\end{ldescription}
\end{Arguments}
\inputencoding{utf8}
\HeaderA{superTargetPODplot}{Generate chemical-wise boxplot of the BMD distributions by super\_target}{superTargetPODplot}
%
\begin{Description}\relax
Generate chemical-wise boxplot of the BMD distributions by super\_target
\end{Description}
%
\begin{Usage}
\begin{verbatim}
superTargetPODplot(
  to.file = F,
  dataset = "heparg2d_toxcast_pfas_pe1_normal_refchems",
  sigset = "screen_large",
  method = "fc",
  celltype = "HepaRG",
  hccut = 0.95,
  tccut = 1.5,
  cutoff = 5
)
\end{verbatim}
\end{Usage}
%
\begin{Arguments}
\begin{ldescription}
\item[\code{to.file}] If TRUE, send the plots to a file

\item[\code{dataset}] Name of the HTTr data set

\item[\code{sigset}] Name of the signature set

\item[\code{method}] Scoring method

\item[\code{celltype}] Name of the cell type

\item[\code{hccut}] Exclude rows in the data set with hitcall less than this value

\item[\code{tccut}] Exclude rows in the data set with top\_over\_cutoff less than this value

\item[\code{cutoff}] The minimum number of signatures hat have to be active in a super
target for the super target to be considered active. Default is 5
\end{ldescription}
\end{Arguments}
\inputencoding{utf8}
\HeaderA{superTargetStats}{Generate hit statistics by super\_target}{superTargetStats}
%
\begin{Description}\relax
Generate hit statistics by super\_target
\end{Description}
%
\begin{Usage}
\begin{verbatim}
superTargetStats(
  do.load = F,
  dataset = "heparg2d_toxcast_pfas_pe1_normal_refchems",
  sigset = "pilot_tiny",
  method = "fc",
  celltype = "HepaRG",
  hccut = 0.95,
  tccut = 1.5,
  cutoff = 5
)
\end{verbatim}
\end{Usage}
%
\begin{Arguments}
\begin{ldescription}
\item[\code{do.load}] If TRUE, Load the large input data file

\item[\code{dataset}] Name of the HTTr data set

\item[\code{sigset}] Name of the signature set

\item[\code{method}] Scoring method

\item[\code{celltype}] Name of the cell type

\item[\code{hccut}] Exclude rows in the data set with hitcall less than this value

\item[\code{tccut}] Exclude rows in the data set with top\_over\_cutoff less than this value

\item[\code{cutoff}] The minimum number of signatures hat have to be active in a super
target for the super target to be considered active. Default is 5
\end{ldescription}
\end{Arguments}
\inputencoding{utf8}
\HeaderA{TxT}{Calculate several statistics on a 2 x 2 matrix}{TxT}
%
\begin{Description}\relax
Calculate several statistics on a 2 x 2 matrix
\end{Description}
%
\begin{Usage}
\begin{verbatim}
TxT(tp, fp, fn, tn, do.p = TRUE, rowname = NA)
\end{verbatim}
\end{Usage}
%
\begin{Arguments}
\begin{ldescription}
\item[\code{tp}] number of true positives

\item[\code{fp}] number of false positives

\item[\code{fn}] number of false negatives

\item[\code{tn}] number of true negatives

\item[\code{do.p}] if TRUE, calcualte an exact p-value

\item[\code{rowname}] if not NA, adda column to the output with this rowname

Returns:
a list of the results
a: TP
b: FP
c: FN
d: TN
sens: sensitivity
spec: specificity
ba: Balanced Accuracy
accuracy: Accuracy
relative.risk: Relative Risk
odds.ratio: Odds Ratio
or.ci.lwr: lower confidence interval of the Odds Ratio
or.ci.upr: upper confidence interval of the Odds Ratio
ppv: Positive Predictive Value
npv: Negative Predictive Value
p.value: Chi-squared p-value
F1: 2TP/(2TP+FP+FN)

sval: All of the results as a tab-delimited string
title: the title of the results as a tab-delimited string
mat: The results as a 1-row data frame
@export
\end{ldescription}
\end{Arguments}
\inputencoding{utf8}
\HeaderA{WRMSE}{Weighted Root-mean-square-error}{WRMSE}
%
\begin{Description}\relax
Computes root-mean-square error with weighted average.
\end{Description}
%
\begin{Usage}
\begin{verbatim}
WRMSE(x, y, w)
\end{verbatim}
\end{Usage}
%
\begin{Arguments}
\begin{ldescription}
\item[\code{x}] First vector of numbers.

\item[\code{y}] Second vector of numbers.

\item[\code{w}] Vector of weights.
\end{ldescription}
\end{Arguments}
%
\begin{Details}\relax
x,y,w should all be the same length. Order of x and y won't change output.
\end{Details}
%
\begin{Value}
Weighted RMSE.
\end{Value}
%
\begin{Examples}
\begin{ExampleCode}
WRMSE(1:3, c(1,3,5), 1:3)
\end{ExampleCode}
\end{Examples}
\printindex{}
\end{document}
